
\chapter{Results and Comparative Analysis} 
\label{ch:Results and Comparative Analysis}
In this chapter an analysis and evaluation of the TimeNET implementation of the multi-trajectory algorithm is presented. First, an analysis of its parameters is conducted and for this purpose a model is designed. Execution of this model with all possible values of different multi-trajectory parameters are shown, and the best combination of parameters is found. Afterwards, a comparative analysis of the implementation with the other three TimeNET implemented simulation algorithms is presented. This analysis is based on a previous research conducted by the author of the current work, \textit{Rare-Event Algorithms Analysis and Simulation} \cite{canabal:rareeventproject}. Its goals were to evaluate RESTART TimeNET implementation and the implemented heuristic for its importance function. The multi-trajectory simulation results are compared together with the results obtained from the earlier work, followed by an analysis of how variations of the model properties exert different impact on the execution time of the algorithms. Finally, a non-Markovian model is analyzed and different simulations executed on it are presented.


\section{Multi-Trajectory Results}
In this section the results from the multi-trajectory simulations are compared with the results from the previous work. The multi-trajectory algorithm has been executed with the default parameters over all variations of the models, but only the most significant results are presented here. 

\input{sourcecode/Tables_Comparative_Results}
